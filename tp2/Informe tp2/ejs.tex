\section{Ejercicios}

\subsection{Ejercicio 1: Read-Write Lock}
Para este ejercicio nos basamos en el libro \textit{"The Little Book of Semaphores" (Seccion 4.2)}. En la misma se analiza y resuelve el problema en el cual varios threads escriben o leen los mismos datos. Pueden leer varios al mismo tiempo, pero solo uno podra escribir.
Al igual que en nuestro Scrabble, queremos que esto suceda para evitar inanicion.
Incluimos como variables un pthread\_mutex\_t llamado \textit{roomEmpty} el cual se activa cuando no hay threads (leyendo o escribiendo) en la seccion critica, y un int \textit{cantidadDeSolicitudes} que son la cantidad de threads leyendo.

Para la lectura modificamos las funciones de la siguiente forma:\\
En el caso del lock

\begin{verbatim}
void RWLock::rlock(){
IF (cantidadDeSolicitudes == 0){			//No habia nadie en la seccion critica
    pthread_mutex_lock(&roomEmpty);		//Bloqueo roomEmpty porque ahora ingresara alguien y dejara de estarlo
}
cantidadDeSolicitudes++;					//Aumento en uno cantidadDeSolicitudes por el nuevo ingresante;
}
\end{verbatim}

Para el unlock
\begin{verbatim}
void RWLock::runlock(){
cantidadDeSolicitudes--;	         //Disminuyo cantidadDeSolicitudes porque uno dejara de leer
IF (!cantidadDeSolicitudes){	              //Si no queda nadie en la seccion critica
    pthread_mutex_unlock(&roomEmpty);        //activo la sen\~al de roomEmpty para el que quiera escribir
	}
}
\end{verbatim}

En el caso de la escritura no debemos permitir que haya mas de un thread en la seccion critica.
 En el caso del lock:
 
\begin{verbatim}
void RWLock::wlock(){
pthread_mutex_lock(&roomEmpty);		//Bloqueamos la variable roomEmpty para que nadie pueda acceder
}
\end{verbatim}

Para el unlock:
\begin{verbatim}
void RWLock::wunlock(){
pthread_mutex_unlock(&roomEmpty);		//Liberamos roomEmpty para que puedan acceder otros
}
\end{verbatim}







\subsection{Ejercicio 2: Backend Multithreaded}
